
\documentclass[a4paper,18pt]{article}
\usepackage{physics}
\usepackage[T2A]{fontenc}
\usepackage[utf8]{inputenc}
\everymath{\displaystyle}
\usepackage{amsmath}


\begin{document}

\title{Задачи по Физике Твердого тела}
\author{Корюн Аванесян}
\date{\today}
\maketitle
 
\textbf{Задача 8.1} Рассмотреть электрон движущийся по круговой орбите вокруг яра с зарядом +е.
Используя формулу для силы Лоренца получить выражение для магнитного поля \textbf{H},
и показать,что круговая частота электрона равна 

\begin{equation}
    \omega = \frac{eH}{2mc} + \left((\frac{eH}{2mc})^2 \pm \frac{e^2}{mr^3}\right)^{1/2}
\end{equation}

 
\textbf{Решение} Общая формула для силы Лоренца есть 

\begin{equation}
    \vec{F} = e(\vec{E} + \frac{1}{c}\vec{v} \times \vec{H})
\end{equation}
Для круговой орбиты имеем следующие формулы 

\begin{equation}
    \begin{aligned}
     \vec{F} &= m(\omega)^2r\vec{n}  \\ 
     \vec{v} \times \vec{H} &= -\omega rH\vec{n} \\
     \vec{E} &= \frac{e}{r^2}\vec{n}
    \end{aligned}
\end{equation}

Подставляя все уравнения (3) в (2), получим

\begin{equation}
    \left(mr\omega^2 + \frac{e}{c}rH\omega - \frac{e^2}{m^2}\right)\vec{n} = 0
\end{equation}

где n - единичный вектор, совпадающий с направлению r. Решая уравнение (4) относительно omega, получаем 

\begin{equation}
    \omega  &=  - \frac{eH}{2mc} \pm \sqrt{(\frac{eH}{2mc})^2 + \frac{e^2}{mr^2}}
\end{equation}

\begin{equation}
  \begin{aligned}
    H \leq 100 \:\:\:\:\:\:\:\:\:\:\:\:
    (\frac{eH}{2mc})^2 \approx 10^24 \:\:\:\:\:\:\:\:\:\:\:\:
    \frac{e^w}{mr^3} \approx 10^32
  \end{aligned}
\end{equation}

Следовательно имеем 

\begin{equation}
    \omega \approx \frac{eH}{2mc} \pm \sqrt{\frac{e^2}{mr^3}}
\end{equation}

Последний член представляет собой угловую скорость при H = 0. Лармовская  частота задается формулой 

\begin{equation}
    \omega_L =  \frac{eH}{2mc}
\end{equation}


\textbf{Задача 8.5} Записать классическую статистическую сумму Z для электронного газа в виде интеграла и показать, что по классической теории магнитная восприимчивость такого газа равна нулю. \\ \\ \\  


\textbf{Решение} Статсумма задается вырождением 

\begin{equation}
    Z = \frac{2}{h^3}\int\int\int{e^{\frac{\sigma(p,q)}{kT}}dxdydzdp_xdp_ydp_z}
\end{equation}

Для свабодного электронного газа \sigma = mv^2/2m, канонический и кинетический импульс связны соотношением
\begin{equation}
    m\vec{v} = \vec{p} - \frac{e}{c}\vec{A}
\end{equation}

Где A векторный потенциал.\\ 
Переходя от канонического к кинетическому, переобразуем элемент фазового пространства

$${dxdydzdp = (J)m^2dv_xdv_ydv_z}$$

Где J - Якобиан

\begin{equation}
    J = \frac{1}{m^3}
    \begin{pmatrix}
        \pdv{p_x}{v_x} & \pdv{p_x}{v_y} & \pdv{p_x}{v_z} \\
        \pdv{p_y}{v_x} & \pdv{p_y}{v_y} & \pdv{p_y}{v_z} \\
        \pdv{p_z}{v_x} & \pdv{p_z}{v_y} & \pdv{p_z}{v_z}
    \end{pmatrix}

\end{equation}


Имея виду что \vec{A} есть функция от H, \vec{A} = rotH, который не зваисит от v_x, v_y, v_z, по этому стат сумма оказывается не зваисящей от H.

\begin{equation}
     Z = \frac{8\pi V}{h^3}\int^{\infty}_{0}{e^{-\frac{(mv)^2}{2mkT}}(mv)^2d(mv) = \frac{2V}{h^3}2\pi mktT}
\end{equation}

Следовательно можем написать что 
\begin{equation}
    M = NkT\pdv{lnZ}{H} = 0
\end{equation}
а поэтому и \kappa = 0, что ставетсвует классическому случаю.

\end{document}

